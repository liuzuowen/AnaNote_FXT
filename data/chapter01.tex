% !TeX root = ../analysisnote.tex

\section{Analysis setup}

This section shows the setup of analysis. 
The first part of this section introduces STAR Fixed Target(FXT) mode. 
Second one is dataset and event selection used in the analysis. 
At last, bad-run rejection and centrality determination are discussed.

%-------------------------------
\subsection{Introduction of FXT}

There are two modes at RHIC-STAR, Fixed Target (FXT) mode and Collider mode. 
FXT mode of STAR covers center of mass energy ($\snn$) from 3 to 13.7 GeV, which extends the flow measurement to lower collision energy.
In this way, FXT mode could reach higher baryon density region ($\mu_B$ = 750 MeV at most). 
To achieve this goal, One piece of gold foil is fixed at the end of Time Projection Chamber (TPC)\cite{PhysRevC.103.034908} as target.
And the projectile which is accelerated gold beam would collide with it. Fig.~\ref{fig:FXTmode} shows the schematic plot for fixed target mode.
\begin{figure}[hbt!]
\centering
\includegraphics[width=0.45\linewidth]{figures/chapter01/FXTmode.png}
\caption{The schematic plot of Fixed Target mode at STAR.}
\label{fig:FXTmode}
\end{figure}

The laboratory system should be transferred to the Center of Mass System (CMS), 
if one want to analysis physical observation conveniently. 
According to the Lorentz transformation, Rapidity (y) of the final state particle in the Center of Mass System could be calculated as equation~\ref{eq:transf_rap}, 
where $y_{lab}$ is rapidity of emitted particle in the laboratory frame, $y_{beam}$ is rapidity of gold beam in the CMS.
And we would discuss flow measurement in the Center of Mass System as follows.
\begin{equation}
    y_{CMS} = y_{lab} - y_{beam}
\label{eq:transf_rap}
\end{equation}

%-------------------------------
\subsection{Dataset and event selection}

In this analysis, four collision energies ($\snn=3, 3.2, 3.5, 3.9$ GeV) from fixed target mode are involved.
The events from minimum bias trigger are selected. Vertex cut in the X and Y direction is applied to exclude the events colliding with beam pipe, 
and vertex cut along the Z direction ensure that selection events are from collision on the gold foil.
The detailed triggers and vertex cuts are summarized in the TABLE.~\ref{tab:dataset}.
\begin{table}
\caption{Dataset and event cuts for $\snn=3, 3.2, 3.5, 3.9$ GeV.}
\label{tab:dataset}
\begin{tabular}{|c|c|c|c|c|c|}
\hline
$\snn$(GeV) & production tag & library tag & trigger ID & Vz(cm)     & Vr (cm) \\ \hline
3.0         & P19ie          & SL20d       & 620052     & [198, 202] & 2       \\ \hline
3.2         & P23id          & SL23d       & 680001     & [198, 202] & 2       \\ \hline
3.5         & P23ie          & SL23e       & 720000     & [198, 202] & 2       \\ \hline
3.9         & P23ie          & SL23e       & 730000     & [198, 202] & 2       \\ \hline
\end{tabular}
\end{table}


%-------------------------------
\subsection{Badrun rejection and centrality determination}

Data production at STAR is run-by-run type, and the data is tagged with different run numbers.
Run-by-run QA is necessary to reject the bad runs, which may caused by the broken detector etc.
This job was finished by STAR QA group, and the bad-run list used in this analysis could be found at the
\href{https://drupal.star.bnl.gov/STAR/pwg/common/bes-ii-run-qa/FXT-datasets}{summary page} from QA group.

Moreover, the centrality determination work is vital in Heavy Ion Collision.
Glauber Monte Carlo (MC) simulation~\cite{KHARZEEV2001121} is applied to describe the reference multiplicity distribution, 
which is corrected by pile-up rejection and luminosity correction etc.
The number of participant particle ($N_part$) and the number of collision particles ($N_coll$)
 can be extracted by the formula~\ref{eq:npart_ncoll} in MC model.
 \begin{equation}
    \frac{d N_{c h}}{d \eta}=n_{p p}\left[(1-x) \frac{N_{p a r t}}{2}+x N_{c o l l}\right]
\label{eq:npart_ncoll}
\end{equation}
This work is finished by STAR Centrality group, a class(\href{https://github.com/star-bnl/star-sw/tree/main/StRoot/StRefMultCorr}{StRefMultCorr}) is provided.
One can call the class and get the refMult cut.


\subsection{Basic track selection}

The basic track cuts can assure every track used in the analysis meet the minimum experimental requirements.
For example, the nHitsFit $ > $ 15 cut requires that the track passing TPC have fifteen fitting hits at least,
which could reduce the track merging effect (two tracks are merged as one.). 
The total basic track cuts are summarized at TABLE~\ref{tab:trackCut}.

\begin{table}
    \centering
    \begin{tabular}{|c|} \hline  
         Track cuts\\ \hline  
         nHitsFit $>$ 15\\ \hline
        DCA $<$ 3 cm\\ \hline
    \end{tabular}
    \caption{The basic track cuts}
    \label{tab:trackCut}
\end{table}